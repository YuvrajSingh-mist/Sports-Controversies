%%
%% This is file `sample-sigconf.tex',
%% generated with the docstrip utility.
%%
%% The original source files were:
%%
%% samples.dtx  (with options: `all,proceedings,bibtex,sigconf')
%% 
%% IMPORTANT NOTICE:
%% 
%% For the copyright see the source file.
%% 
%% Any modified versions of this file must be renamed
%% with new filenames distinct from sample-sigconf.tex.
%% 
%% For distribution of the original source see the terms
%% for copying and modification in the file samples.dtx.
%% 
%% This generated file may be distributed as long as the
%% original source files, as listed above, are part of the
%% same distribution. (The sources need not necessarily be
%% in the same archive or directory.)
%%
%%
%% Commands for TeXCount
%TC:macro \cite [option:text,text]
%TC:macro \citep [option:text,text]
%TC:macro \citet [option:text,text]
%TC:envir table 0 1
%TC:envir table 0 1
%TC:envir tabular [ignore] word
%TC:envir displaymath 0 word
%TC:envir math 0 word
%TC:envir comment 0 0
%%
%% The first command in your LaTeX source must be the \documentclass
%% command.
%%
%% For submission and review of your manuscript please change the
%% command to \documentclass[manuscript, screen, review]{acmart}.
%%
%% When submitting camera ready or to TAPS, please change the command
%% to \documentclass[sigconf]{acmart} or whichever template is required
%% for your publication.
%%
%%
%\documentclass[sigconf]{acmart}
\documentclass[sigconf, review]{acmart}
%%
%% \BibTeX command to typeset BibTeX logo in the docs
\AtBeginDocument{%
  \providecommand\BibTeX{{%
    Bib\TeX}}}

%% Rights management information.  This information is sent to you
%% when you complete the rights form.  These commands have SAMPLE
%% values in them; it is your responsibility as an author to replace
%% the commands and values with those provided to you when you
%% complete the rights form.
\setcopyright{acmlicensed}
\copyrightyear{2018}
\acmYear{2018}
\acmDOI{XXXXXXX.XXXXXXX}
%% These commands are for a PROCEEDINGS abstract or paper.
\acmConference[Conference acronym 'XX]{Make sure to enter the correct
  conference title from your rights confirmation email}{June 03--05,
  2018}{Woodstock, NY}
%%
%%  Uncomment \acmBooktitle if the title of the proceedings is different
%%  from ``Proceedings of ...''!
%%
%%\acmBooktitle{Woodstock '18: ACM Symposium on Neural Gaze Detection,
%%  June 03--05, 2018, Woodstock, NY}
\acmISBN{978-1-4503-XXXX-X/2018/06}


%%
%% Submission ID.
%% Use this when submitting an article to a sponsored event. You'll
%% receive a unique submission ID from the organizers
%% of the event, and this ID should be used as the parameter to this command.
%%\acmSubmissionID{123-A56-BU3}

%%
%% For managing citations, it is recommended to use bibliography
%% files in BibTeX format.
%%
%% You can then either use BibTeX with the ACM-Reference-Format style,
%% or BibLaTeX with the acmnumeric or acmauthoryear sytles, that include
%% support for advanced citation of software artefact from the
%% biblatex-software package, also separately available on CTAN.
%%
%% Look at the sample--biblatex.tex files for templates showcasing
%% the biblatex styles.
%%

%%
%% The majority of ACM publications use numbered citations and
%% references.  The command \citestyle{authoryear} switches to the
%% "author year" style.
%%
%% If you are preparing content for an event
%% sponsored by ACM SIGGRAPH, you must use the "author year" style of
%% citations and references.
%% Uncommenting
%% the next command will enable that style.
%%\citestyle{acmauthoryear}

\usepackage{graphicx}
\usepackage{amsmath, amssymb}
\usepackage{hyperref}
%\usepackage{cite}
\usepackage{algorithm}
\usepackage{algorithmic}
\usepackage{hyperref}
\usepackage{caption}
\usepackage{graphicx}
\graphicspath{ {./images/} }
\usepackage{booktabs} % For better table formatting
\usepackage{multirow} % For multirow cells
\usepackage{array} % For custom column alignment
\usepackage{tabularx} % For auto-adjusting table width
\usepackage{booktabs} % For better table formatting
\usepackage{adjustbox} % For scaling the table
\usepackage{geometry} % For better page margins
\usepackage{array}
\usepackage[section]{placeins}
\usepackage[section]{placeins} % for \FloatBarrier
%%
%% end of the preamble, start of the body of the document source.
\begin{document}

%%
%% The "title" command has an optional parameter,
%% allowing the author to define a "short title" to be used in page headers.
%\title{The Name of the Title Is Hope}
\title{SportsOri: A Novel Dataset for Analyzing Public Sentiment on Controversial Sports Events in YouTube Comments}

%%
%% The "author" command and its associated commands are used to define
%% the authors and their affiliations.
%% Of note is the shared affiliation of the first two authors, and the
%% "authornote" and "authornotemark" commands
%% used to denote shared contribution to the research.
% \author{Ben Trovato}
% \authornote{Both authors contributed equally to this research.}
% \email{trovato@corporation.com}
% \orcid{1234-5678-9012}
% \author{G.K.M. Tobin}
% \authornotemark[1]
% \email{webmaster@marysville-ohio.com}
% \affiliation{%
%   \institution{Institute for Clarity in Documentation}
%   \city{Dublin}
%   \state{Ohio}
%   \country{USA}
% }

% \author{Lars Th{\o}rv{\"a}ld}
% \affiliation{%
%   \institution{The Th{\o}rv{\"a}ld Group}
%   \city{Hekla}
%   \country{Iceland}}
% \email{larst@affiliation.org}

% \author{Valerie B\'eranger}
% \affiliation{%
%   \institution{Inria Paris-Rocquencourt}
%   \city{Rocquencourt}
%   \country{France}
% }

% \author{Aparna Patel}
% \affiliation{%
%  \institution{Rajiv Gandhi University}
%  \city{Doimukh}
%  \state{Arunachal Pradesh}
%  \country{India}}

% \author{Huifen Chan}
% \affiliation{%
%   \institution{Tsinghua University}
%   \city{Haidian Qu}
%   \state{Beijing Shi}
%   \country{China}}

% \author{Charles Palmer}
% \affiliation{%
%   \institution{Palmer Research Laboratories}
%   \city{San Antonio}
%   \state{Texas}
%   \country{USA}}
% \email{cpalmer@prl.com}

% \author{John Smith}
% \affiliation{%
%   \institution{The Th{\o}rv{\"a}ld Group}
%   \city{Hekla}
%   \country{Iceland}}
% \email{jsmith@affiliation.org}

% \author{Julius P. Kumquat}
% \affiliation{%
%   \institution{The Kumquat Consortium}
%   \city{New York}
%   \country{USA}}
% \email{jpkumquat@consortium.net}


\author{Yuvraj Singh}
%\authornote{Both authors contributed equally to this research.}
% \email{trovato@corporation.com}
% \orcid{1234-5678-9012}
% \author{G.K.M. Tobin}
% \authornotemark[1]

\affiliation{%
  \institution{IIIT Bhubaneswar}
  \city{Bhubaneswar, Orissa}
  %\state{Ohio}
  \country{India}
}
\email{yuvraj.mist@gmail.com}

\author{Devadripta Jadhav}
\affiliation{%
  \institution{Savitribai Phule Pune University}
  \city{Pune, Maharashtra}
  \country{India}}
\email{devadripta@gmail.com}

\author{Samiksha Boduwar}
\affiliation{%
  \institution{Savitribai Phule Pune University}
  \city{Pune, Maharashtra}
  \country{India}}
\email{samikshaboduwar@gmail.com}




\author{Kripabandhu Ghosh}
\affiliation{%
  \institution{IISER Kolkata}
  \city{Mohanpur, West Bengal}
  \country{India}}
\email{kripaghosh@iiserkol.ac.in}
%%
%% By default, the full list of authors will be used in the page
%% headers. Often, this list is too long, and will overlap
%% other information printed in the page headers. This command allows
%% the author to define a more concise list
%% of authors' names for this purpose.
\renewcommand{\shortauthors}{Singh et al.}

%%
%% The abstract is a short summary of the work to be presented in the
%% article.
\begin{abstract}
  This paper presents an analysis of YouTube comments on famous and controversial Public Sports Events. We explore public stance (stance detection) on a total of 6 famous controversial sports incidents by extracting and processing YouTube comments. Stance detection is performed on multiple events, including \textit{The Underarm Incident}, \textit{Jonny Bairstow's Run-Out} Incident, \textit{Ashwin's Mankadding} Event, \textit{Luis Suarez Handball} Event etc. 
The complete event details,  results and evaluation metrics will be discussed in detail in subsequent sections.
\end{abstract}

%%
%% The code below is generated by the tool at http://dl.acm.org/ccs.cfm.
%% Please copy and paste the code instead of the example below.
%%
% \begin{CCSXML}
% <ccs2012>
%  <concept>
%   <concept_id>00000000.0000000.0000000</concept_id>
%   <concept_desc>Do Not Use This Code, Generate the Correct Terms for Your Paper</concept_desc>
%   <concept_significance>500</concept_significance>
%  </concept>
%  <concept>
%   <concept_id>00000000.00000000.00000000</concept_id>
%   <concept_desc>Do Not Use This Code, Generate the Correct Terms for Your Paper</concept_desc>
%   <concept_significance>300</concept_significance>
%  </concept>
%  <concept>
%   <concept_id>00000000.00000000.00000000</concept_id>
%   <concept_desc>Do Not Use This Code, Generate the Correct Terms for Your Paper</concept_desc>
%   <concept_significance>100</concept_significance>
%  </concept>
%  <concept>
%   <concept_id>00000000.00000000.00000000</concept_id>
%   <concept_desc>Do Not Use This Code, Generate the Correct Terms for Your Paper</concept_desc>
%   <concept_significance>100</concept_significance>
%  </concept>
% </ccs2012>
% \end{CCSXML}

% \ccsdesc[500]{Do Not Use This Code~Generate the Correct Terms for Your Paper}
% \ccsdesc[300]{Do Not Use This Code~Generate the Correct Terms for Your Paper}
% \ccsdesc{Do Not Use This Code~Generate the Correct Terms for Your Paper}
% \ccsdesc[100]{Do Not Use This Code~Generate the Correct Terms for Your Paper}

% %%
% %% Keywords. The author(s) should pick words that accurately describe
% %% the work being presented. Separate the keywords with commas.
% \keywords{Do, Not, Us, This, Code, Put, the, Correct, Terms, for,
%   Your, Paper}
\begin{CCSXML}
<ccs2012>
   <concept>
       <concept_id>10002951.10003260.10003277</concept_id>
       <concept_desc>Information systems~Web mining</concept_desc>
       <concept_significance>500</concept_significance>
       </concept>
 </ccs2012>
\end{CCSXML}

\ccsdesc[500]{Information systems~Web mining}
%%
%% Keywords. The author(s) should pick words that accurately describe
%% the work being presented. Separate the keywords with commas.
\keywords{Stance Detection, YouTube Comments, Social Media Analysis, Sports Controversies}
%% A "teaser" image appears between the author and affiliation
%% information and the body of the document, and typically spans the
%% page.
% \begin{teaserfigure}
%   \includegraphics[width=\textwidth]{sampleteaser}
%   \caption{Seattle Mariners at Spring Training, 2010.}
%   \Description{Enjoying the baseball game from the third-base
%   seats. Ichiro Suzuki preparing to bat.}
%   \label{fig:teaser}
% \end{teaserfigure}

% \received{20 February 2007}
% \received[revised]{12 March 2009}
% \received[accepted]{5 June 2009}

%%
%% This command processes the author and affiliation and title
%% information and builds the first part of the formatted document.
\maketitle

\section{Introduction}

%Hello~\cite{sportqa}


% ACM's consolidated article template, introduced in 2017, provides a
% consistent \LaTeX\ style for use across ACM publications, and
% incorporates accessibility and metadata-extraction functionality
% necessary for future Digital Library endeavors. Numerous ACM and
% SIG-specific \LaTeX\ templates have been examined, and their unique
% features incorporated into this single new template.

% If you are new to publishing with ACM, this document is a valuable
% guide to the process of preparing your work for publication. If you
% have published with ACM before, this document provides insight and
% instruction into more recent changes to the article template.

% The ``\verb|acmart|'' document class can be used to prepare articles
% for any ACM publication --- conference or journal, and for any stage
% of publication, from review to final ``camera-ready'' copy, to the
% author's own version, with {\itshape very} few changes to the source.

Sports engages billions of followers worldwide\footnote{\url{https://www.statista.com/chart/14329/global-interest-in-football/}} 
and impacts the economy \cite{sportseconomics20221}. Sports controversies often ignite passionate discussions among fans, analysts, and players. With the rise of social media, platforms like YouTube have become central to these discussions. This study aims to analyze the stances or perform opinion mining namely for, against, and neutral on comments from famous social media platforms -- YouTube, focusing on events such as Jonny Bairstow's Run-Out Incident, Luis Suarez Handball Event etc.
To our knowledge, the first-ever study of civic engagement in controversial sports events (cricket and football) spans around 40 years. LLMs (Llama3 family) were used for initial annotations (stance) of comments and later fine-tuned for comparative performance analysis ($~$30\% boost in accuracy).

% Sports controversies often ignite passionate discussions among fans, analysts, and players. With the rise of social media, platforms like YouTube have become central to these discussions. This study aims to analyze the stances or perform opinion mining namely for, against, and neutral on comments from famous social media platforms like YouTube, focusing on events such as Jonny Bairstow's Run-Out Incident, Luis Suarez Handball Event etc.
% To our knowledge, the first-ever study of civic engagement in controversial sports events (cricket and football) spans around 40 years. LLMs (Llama3 family) were used for initial annotations (stance) of comments and later fine-tuned for comparative performance analysis ($~$30\% boost in accuracy).%~\cite{Lamport:LaTeX}

Our study stands apart from its counterparts by focusing on public sentiment analysis surrounding controversial sports events, specifically through the lens of YouTube comments. Papers like SportQA \cite{xia-etal-2024-sportqa} aims to evaluate how well large language models (LLMs) understand sports knowledge through a benchmark dataset while Run Like a Girl! \cite{harrison-etal-2023-run} delves into gender bias in sports-related datasets, highlighting underrepresentation and naming disparities, we shift the focus to how people react to contentious moments in sports. It uses stance detection techniques to analyze public opinion, offering insights into the emotional and polarized responses to events like the Underarm Incident or Jonny Bairstow’s Run-Out. Moreover, studies like Generating Sports News from Live Commentary \cite{huang-etal-2020-generating} are centred on automating sports news generation from live commentary, emphasizing summarization and natural language generation. In essence, while the other studies explore sports understanding, bias, and news automation, our study uniquely examines the social media-driven public discourse around sports controversies, making it distinct in its focus on human reactions and sentiment rather than dataset creation, model evaluation, or bias analysis. Thus, after a thorough human verification, we are releasing a dataset of 40K+ opinion labelled comments (Section \ref{sec:dataset}) and discuss the results in Section \ref{sec:res_discuss}. 

% The Frank Lampard "Ghost Goal" during the 2010 World Cup against Germany was a pivotal moment of controversy. The goal was clearly visible on replays as having crossed the line, but referee Jorge Larrionda and assistant Mauricio Espinosa missed it, leading to widespread media coverage and fan outrage 12. This incident was a significant catalyst for the introduction of goal-line technology in football, highlighting the need for technological assistance in critical refereeing decisions35. The decision was seen as a turning point in the match, which England eventually lost 4-12. The controversy overshadowed England's overall performance at the tournament, shifting focus from managerial issues to the refereeing error2.

%\cite{sportqa}
%{\color{red}KRIPA: Cite these https://aclanthology.org/search/?q=sports and explain why our work is different}


\section{Dataset Creation: SportsOpi}\label{sec:dataset}



\subsection{Data Collection}

We first identified famous public sports controversies by randomly picking 100 such events (football and cricket) from Wikipedia \footnote{\url{https://en.wikipedia.org/wiki/Category:Sports_controversies}} 
 and fetching 100 YouTube videos, sorted by "Most relevant" filter, related to each such controversy. Subsequently, the comments were fetched through YouTube Data API \footnote{\url{https://developers.google.com/youtube/v3}}, and sorted in decreasing order of number of comments. Thus, the final controversial events were chosen by considering the total number of samples each event had followed by quality, engagement and balance of polarity. Our methodology focussed on the creation of a curated playlist for each event, ensuring diverse opinions. Since we are looking for public engagement on historic sports controversies, we look for events that have opinions of diverse classes viz. {\bf Favour}, {\bf Against}, {\bf Neutral} and {\bf Irrelevant}. The respective definitions for these labels can be found here.

% The data extraction process involved:
% \begin{enumerate}
%     %\item Identifying a famous public sports controversy from relevant sources like YouTube, Wikipedia and news sources based on quality and engagement.
%     %\item Identifying relevant videos for the chosen public sports controversies.
%     \item Extracting comments using the YouTube Data API for each of the identified videos.
%     \item Sorting the videos by the number of comments. Selected the top 50 (average) with the highest amount of comments.
%     \item Stored the comments for the top 50 videos in a structured CSV format.
%     \item Repeated the process until we had such data for 10 such controversies.
% \end{enumerate}


% \begin{table}[h]
%     \centering
%     \renewcommand{\arraystretch}{0.7} % Increase row height
%     \resizebox{0.5\textwidth}{!}{ % Increase table width beyond text width
%     \begin{tabular}{|l|r|} 
%         \hline
%         \textbf{Controversy Name} & \textbf{Number of Comments} \\ 
%         \hline
        
%         Ashwin Mankading  & 3785  \\  
%         % \hline
%         Frank Lampard Ghost Goal & 13520 \\
%         % \hline
%         Johny Bairstow Runout & 6073\\
%         % \hline
%         Luis Suarez Handball & 11546 \\
%         % \hline
%         Maradona Hand of God & 5159 \\
%         % \hline
%         The Underarm Incident & 3676\\
%         % \hline
%         \hline
%     Total & 43759\\
%     \hline
%     \end{tabular}
%     }

%    %\captionsetup{justification=centering, font=medium} % Increase font size & center
%     \caption{Name of the controversies used and their number of comments. {\color{red}KRIPA: class-wise nos also}}
%     \label{tab:controversies}
% \end{table}


% \begin{table}[h]
%     \centering
%     \renewcommand{\arraystretch}{0.8} % Adjust row height
%     \resizebox{0.5\textwidth}{!}{ % Auto resize table width
%     \begin{tabular}{|l|r|r|r|r|} 
%         \hline
%         \textbf{Controversy Name} & \textbf{Favor} & \textbf{Neutral} & \textbf{Irrelevant} & \textbf{Against} \\ 
%         \hline
%         Maradona Hand of God  & 2100 & 900 & 1500 & 500  \\  
%         Frank Lampard Ghost Goal & 7000 & 1800 & 1100 & 3200 \\
%         Luis Suarez Handball & 2400 & 2200 & 4200 & 2600 \\
%         \hline
%     \end{tabular}
%     }
%     \caption{Class-wise distribution of comments for different controversies.}
%     \label{tab:controversy_distribution}
% \end{table}




\subsection{The Events}

Following the data collection pipeline, a total of six events were chosen, namely, \noindent {\bf Frank Lampard Ghost Goal \footnote{\url{https://thesefootballtimes.co/2016/02/28/diego-maradona-and-the-reality-behind-the-hand-of-god/}}}, \noindent {\bf Maradona Hand of God \footnote{\url{https://thesefootballtimes.co/2016/02/28/diego-maradona-and-the-reality-behind-the-hand-of-god/}} }, {\bf Luis Suárez's deliberate handball \footnote{\url{https://www.skysports.com/football/news/12040/12759389/uruguays-luis-suarez-says-he-will-not-apologise-to-ghana-for-his-handball-that-knocked-them-out-of-2010-world-cup}} }, {\bf The Jonny Bairstow Ashes Runout \footnote{\url{https://www.espncricinfo.com/story/jonny-bairstow-reignites-ashes-stumping-row-1405100}}}, {\bf Ravichandran Ashwin's Mankading \footnote{\url{https://www.espncricinfo.com/story/jonny-bairstow-reignites-ashes-stumping-row-1405100}}}, {\bf The Underarm \footnote{\url{https://www.espncricinfo.com/story/trevor-chappell-s-underarm-delivery-498574}}}.

% Frank Lampard's "Ghost Goal" in the 2010 World Cup against Germany was a pivotal moment of controversy. The goal, clearly crossing the line, was not awarded by referee Jorge Larrionda and assistant Mauricio Espinosa, who was distracted at the time \footnote{\url{https://www.givemesport.com/88026866-frank-lampard-ghost-goal-how-did-officials-miss-england-stars-shot-crossing-the-line/}}. This incident was widely covered in media and sparked calls for goal-line technology, later implemented by FIFA. The decision contributed to Germany's victory with 4-1, overshadowing England's overall performance in the tournament. The incident remains a significant moment in football history, symbolizing the need for technological advancements in refereeing.

% : During the quarter-final match between Argentina and England, Maradona used his hand\footnote{\url{https://thesefootballtimes.co/2016/02/28/diego-maradona-and-the-reality-behind-the-hand-of-god/}} to score a goal. The incident was set against the backdrop of the Falklands War, adding a layer of nationalistic fervour to the match. The goal sparked widespread media coverage and fan outrage, with many criticizing Maradona's actions as unsportsmanlike. Maradona later attributed the goal to the "Hand of God," a phrase that has become synonymous with the event. The controversy highlighted the need for technological advancements in refereeing, later implemented by FIFA. Years later, Maradona acknowledged the handball, viewing it as "symbolic revenge" for Argentina's defeat in the Falklands War.

% % Handball

% \footnote{\url{https://www.skysports.com/football/news/12040/12759389/uruguays-luis-suarez-says-he-will-not-apologise-to-ghana-for-his-handball-that-knocked-them-out-of-2010-world-cup}} in the 2010 World Cup quarter-final against Ghana was a pivotal moment of controversy. Suárez used his hands to block Dominic Adiyiah's goal-bound shot, earning a red card and a penalty for Ghana. However, Asamoah Gyan missed the penalty, and Uruguay won the match in a penalty shootout, denying Ghana a historic semi-final spot. The incident sparked widespread outrage in Ghana, with many viewing Suárez's actions as unsportsmanlike. Suárez has consistently refused to apologize for the handball, stating that he would only apologize if he had injured a player, not for a tactical decision that led to a penalty. The event remains a significant point of contention between Uruguay and Ghana, with Suárez often being seen as a villain in Ghanaian eyes.


%  \footnote{\url{https://www.espncricinfo.com/story/jonny-bairstow-reignites-ashes-stumping-row-1405100}} runout during the Ashes 2023 was a contentious moment that sparked significant debate about the spirit of cricket. Bairstow was dismissed by Alex Carey after leaving his crease prematurely, a move that was within the rules but questioned by many for its sportsmanship. The incident led to heated discussions among former cricketers and pundits, with some defending the dismissal as legitimate and others viewing it as against the spirit of the game. England captain Joe Root initially expressed anger but later acknowledged the legality of the dismissal, emphasizing the importance of players staying within their crease. The incident ignited strong reactions from fans and players alike, with some members of the Marylebone Cricket Club facing disciplinary actions for their behaviour.



%  \footnote{\url{https://www.espncricinfo.com/story/if-you-think-it-doesn-t-fit-into-cricket-change-the-rule-ashwin-on-mankading-1180028}} of Jos Buttler in the 2019 Indian Premier League (IPL) match between Kings XI Punjab and Rajasthan Royals sparked significant controversy. Ashwin removed the bails at the non-striker's end after noticing Buttler had left the crease before the ball was bowled, a move that was legal but criticized for being against the spirit of the game. The incident led to intense debate, with some defending Ashwin's actions as legitimate under the rules and others accusing him of poor sportsmanship123. The controversy reignited discussions about the "Mankad" dismissal, named after Vinoo Mankad, who first used this method in a 1947-48 Test match against Australia. Ashwin defended his actions, stating that players should stay within their crease to avoid such dismissals



% \footnote{\url{https://www.espncricinfo.com/story/trevor-chappell-s-underarm-delivery-498574}} Incident in 1981 was a highly controversial moment in cricket history. During an ODI match between Australia and New Zealand at the Melbourne Cricket Ground, Australian captain Greg Chappell instructed his brother Trevor Chappell to bowl an underarm delivery to prevent New Zealand from hitting a six off the final ball, which would have tied the match. This action, although legal at the time, was widely criticized as being against the spirit of the game. The incident sparked significant outrage in New Zealand and Australia, with many viewing it as unsportsmanlike. It led to an amendment in the laws of cricket, banning underarm deliveries in one-day matches as they were deemed "not within the spirit of the game". The incident remains a notable example of how actions within the rules can still violate the game's ethos, and it continues to be remembered as one of the most infamous moments in cricket history.


%{\color{red} First write about the six events in brief and state why did you choose these events (they had many comments, they span across a long duration of time, are from two very popular sports in the world and cover multiple continents.}


A summarized version of the extraction code is available, and the full code repository link will be available at our \href{https://github.com/YuvrajSingh-mist/Public-Sports-Controversy/tree/master}{{\bf Github repository}}.


% \begin{figure}[h]
%     \centering
    %\begin{minipage}{0.48\textwidth}
        % \centering
        % \includegraphics[width=0.5\textwidth]{tableonechart.jpg} % Adjust width as needed
        % \caption{Number of \textit{Favor}, \textit{Neutral} and \textit{Against} labels}
        % \label{fig:labels}
    %\end{minipage}\hfill
%     \begin{minipage}{0.48\textwidth}
%         \centering
%         \includegraphics[width=\textwidth, keepaspectratio]{praise_and_criticism.jpg} % Adjust width as needed
%         \caption{Comparison of Types of Praise (Favor) and Criticism (Against) for a sample of 200 comments.}
%         \label{fig:praise_criticism}
%     \end{minipage}
 % \end{figure}


% \vspace{5mm} % Add space before table
% \begin{figure}[htbp]
%     \centering
%     \includegraphics[width=0.8\textwidth, keepaspectratio]{tableonechart.jpg} % Adjust width as needed
%     \caption{Number of \textit{Favor}, \textit{Neutral} and \textit{Against} labels}
%     \label{fig:events}
% \end{figure}

% \vspace{5mm} % Add space before table
% \begin{figure}[htbp]
%     \centering
%     \includegraphics[width=0.8\textwidth, keepaspectratio]{praise_and_criticism.jpg} % Adjust width as needed
%     \caption{Comparison of Types of Praise (\textit{Favor}) and Criticism (\textit{Against}) for a sample of 200 comments.}
%     \label{fig:events}
% \end{figure}





\begin{figure}[h]
    \centering
    %\begin{minipage}{0.48\textwidth}
        \centering
        \includegraphics[width=0.5\textwidth]{SportsOpi.jpg} % Adjust width as needed
        \caption{Opinion Annotation  + Data Collection pipeline}
        \label{fig:labels}
    %\end{minipage}\hfill
%     \begin{minipage}{0.48\textwidth}
%         \centering
%         \includegraphics[width=\textwidth, keepaspectratio]{praise_and_criticism.jpg} % Adjust width as needed
%         \caption{Comparison of Types of Praise (Favor) and Criticism (Against) for a sample of 200 comments.}
%         \label{fig:praise_criticism}
%     \end{minipage}
 \end{figure}



% \subsection{Data Processing}
% The extracted comments were preprocessed as follows - 

% \begin{enumerate}
%  \item Removed special characters, stopwords, and non-English comments. We focussed on the evaluation of mostly English content. 
%  \item Irrelevant columns such as nested replies, time of comment and other metadata were removed. 
% \item Additional cleaning steps included normalization and duplicate removal, which were essential to enhance the accuracy of the subsequent sentiment and stance analysis.

% \end{enumerate}

% Custom footer settings
% \fancyfoot[C]{This is a centered footer}








% \begin{table}[h]
%     \centering
%     \renewcommand{\arraystretch}{0.7} % Increase row height
%     \resizebox{0.5\textwidth}{!}{ % Increase table width beyond text width
%     \begin{tabular}{|l|r|} 
%         \hline
%         \textbf{Controversy Name} & \textbf{Number of Comments} \\ 
%         \hline
        
%         Ashwin Mankading  & 3785  \\  
%         % \hline
%         Frank Lampard Ghost Goal & 13520 \\
%         % \hline
%         Johny Bairstow Runout & 6073\\
%         % \hline
%         Luis Suarez Handball & 11546 \\
%         % \hline
%         Maradona Hand of God & 5159 \\
%         % \hline
%         The Underarm Incident & 3676\\
%         % \hline
%         \hline
%     Total & 43759\\
%     \hline
%     \end{tabular}
%     }

%    %\captionsetup{justification=centering, font=medium} % Increase font size & center
%     \caption{Name of the controversies used and their number of comments. {\color{red}KRIPA: class-wise nos also}}
%     \label{tab:controversies}
% \end{table}


\subsection{Opinion Annotation Pipeline}

\textbf{Figure 1} shows the process of our annotation pipeline for opinion mining.
The process for stance detection on our curated dataset consists of two stages as follows - 

\noindent {\bf Stage I:} After the \textbf{Data Collection Pipeline}, a dataset of comments from the chosen 6 sports controversies was created from which we sampled 200 random comments on which initial annotation was done using zero-shot prompt using Llama 3.1-8b Instruct model. 
LLMs were used as initial annotators \cite{tan2024largelanguagemodelsdata, pavlovic-poesio-2024-effectiveness} due to the popularity and success of the same in the synthetic data generation domain. 
Subsequently, we, to do the initial annotation ( \textit{Favor}, \textit{Against}, \textit{Neutral}, \textit{Irrelevant}).
It is important to keep in mind that the opinion labels were made with the subject of the controversy in mind, like Maradona from \textit{Maradona Hand of God} etc. We did the human verification with this idea in mind \footnote{\url{https://github.com/YuvrajSingh-mist/Public-Sports-Controversy/tree/master/data/PDFs}}. \\
{\bf Stage II:} Next, we humanely verified the labels as a result of {\bf Stage-I}. Thereafter, a sample of 20 comments were chosen, which became the basis of the few-shot prompt \footnote{\url{https://github.com/YuvrajSingh-mist/Public-Sports-Controversy/tree/master/data/Prompts}}. Only a few samples were used, since LLMs are prone to "overfit" to a specific type of data/samples provided in the k-shot prompt (if in excess).

This few-shot prompt was then used to annotate the entire dataset of comments with opinions. This was followed by thorough human verification.

\textbf{Table 1} shows the total number of comments, segregated into class-wise number of samples as well.
\textbf{Table 2 shows the result of our annotation process. It consists of sample of each of the four labels from our dataset.}
% \begin{enumerate}

% \item
%     \textbf{Stage-I} (Preparation of "Gold Standard" labels): 
%     \begin{enumerate}
%         \item \textbf{First pass} - In our first pass, we used an open-sourced LLM from HuggingFace, namely Llama 3.1-8b-Instruct. Custom prompts \footnote{\url{https://github.com/YuvrajSingh-mist/Public-Sports-Controversy/blob/master/data/prompts.json}} were constructed for each of the controversies with clear instructions to perform stance detection (for, against, neutral and irrelevant) on a random sample of 200 comments for each of these controversies (zero-shot prompt). 
%         \item \textbf{Second pass} - Subsequently, out of the 200 annotated examples from the \textbf{First pass}, the labels were humanely verified to account for misclassification and a set of 20 such verified labels were taken to form a few shot prompt to annotate the dataset. and/or if further tuning of the prompt was needed to better the distribution of the data.
%         \item \textbf{Third Pass} - The tuned few shot prompts were used to perform the stance detection on our dataset and the final labels were generated thereafter.
        
%     \end{enumerate}

% \item
%     \textbf{Stage-II} (Fine Tuning of LLMs on the stance-labeled dataset) : 
%     \begin{enumerate}

%       \item We used models from the Unsloth library which provides ~70\% reduction in memory usage and up to 2x inference speed.
%       \item We chose models like Llama 3.1-8b-Instruct model (same as used for labelling the dataset) to account for the difference made in the metrics when the same model is to be fined-tuned on our dataset.


%     \\

%     \item Fine tuning on our dataset led to an average increment of ~30\% in terms of accuracy measured and drastic improvement in metrics such as F1, precision and recall as shown by confusion matrix and classification reports compared to its non-fine-tuned model(no data balancing techniques were employed).
%     \end{enumerate}

% % \\
% \end{enumerate}




% \begin{table}[htbp]
% \centering
% \caption{Comparison of Events for Maradona, Frank Lampard, and Luis Suarez}
% \label{tab:events}
% \begin{adjustbox}{width=\textwidth} % Adjust table width to fit the page
% \begin{tabularx}{\textwidth}{l {6}{X}} % X columns auto-adjust width
% \toprule
% \textbf{Event} & \textbf{Direct Criticism} & \textbf{Direct Praise} & \textbf{Indirect Criticism} & \textbf{Indirect Praise} & \textbf{Slang Usage} & \textbf{Racial Abuse} \\
% \midrule
% \textbf{Maradona} & 12 & 45 & 23 & 34 & 10 & 5 \\
% \textbf{Frank Lampard} & 8 & 50 & 18 & 40 & 7 & 3 \\
% \textbf{Luis Suarez} & 15 & 30 & 25 & 28 & 12 & 8 \\
% \bottomrule
% \end{tabularx}
% \end{adjustbox}
% \end{table}


The following details the above-mentioned pipeline for each of the controversies used to constitute our dataset {\color{red}KRIPA: there should NOT be separate strategies for different events. If it is done, it needs to be justified}. - 

\begin{table}[h]
    \centering
    \small
    % \renewcommand{\arraystretch}{0.5} % Adjust row height
    % \setlength{\tabcolsep}{4pt}% Adjust column separation locally
    % \resizebox{\textwidth}{!}{ % Adjust table width to fit the page
    {
    \begin{tabular}{|c|c|c|c|c|c|} 
        \hline
        \textbf{Event} & \textbf{\#C} & \textbf{F} & \textbf{N} & \textbf{I} & \textbf{A} \\ 
        \hline
        Ashwin Mankading  & 3785  & 205 & 414 & 1734 & 1424 \\  
        Frank Lampard Ghost Goal & 13520 & 7000 & 1800 & 1100 & 3200 \\
        Johny Bairstow Runout & 6073 & 331 & 1936 & 1786 & 1987 \\
        Luis Suarez Handball & 11546 & 2400 & 2200 & 4200 & 2600 \\
        Maradona Hand of God & 5159 & 2100 & 900 & 1500 & 500 \\
        The Underarm Incident & 3676 & 330 & 126 & 1063 & 2113 \\
        \hline
        \textbf{Total} & \textbf{43759} & \textbf{12336} & \textbf{7376} & \textbf{11356} & \textbf{11824} \\
        \hline
    \end{tabular}
    }
    \caption{Name of comments, and class-wise distribution of comments.}
    \label{tab:merged_controversies}
\end{table}

\section{Results and Discussion}\label{sec:res_discuss}

Preliminary analysis indicates a significant division in public opinion across the six events within our dataset. 
Fine Tuning on our dataset improves the accuracy of the labels by a drastic margin along with other metrics such as F1 score, recall and precision as compared to the base instruct model.

\textbf{Table 3} shows the result of fine-tuning models on each of the six events.
Detailed results, including the distribution of stances (For, Against, Neutral, Irrelevant) and evaluation metrics (accuracy, precision, recall, F1-score).

% \subsubsection{The Underarm Incident}
% For the Underarm Incident, we employed a fine-tuned LLaMA-3 model from Unsloth. A structured prompt was used to classify comments into four categories: \textbf{For, Against, Neutral,} and \textbf{Irrelevant}. The responses were returned in a JSON format and then parsed to extract the stance label and the underlying rationale.

% We followed the above-mentioned process for the \textit{Maradona Hand of God} event, \textit{the Luis Suarez Handball} Event and the \textit{Frank Lampard Ghost Goal} Event.
% \\

% \subsubsection{Jonny Bairstow's Run-Out and Ashwin's Mankadding Events Using the OLLAMA Framework}
% For Jonny Bairstow's Run-Out Incident and Ashwin's Mankadding Event, we utilized the OLLAMA framework. Detailed API requests were sent with prompts explaining the context of each event, and JSON responses were parsed to extract the stance label and associated reason. 

% % The full code details for this approach are available in the repository: \texttt{[Insert Repository Link]}.


%\vspace{5mm} % Add space before table

% In your preamble:



% Wherever you want the table:
%\FloatBarrier % ensures all previous floats are processed
\begin{table*}[!htbp]
    \centering
    { % Start local grouping so formatting changes do not leak
      \renewcommand{\arraystretch}{1.2}% Adjust row height locally
      \setlength{\tabcolsep}{4pt}% Adjust column separation locally
      \begin{tabular}{|c|p{4cm}|p{4cm}|p{4cm}|}
          \hline
          \textbf{Event} & \textbf{Favor} & \textbf{Against} & \textbf{Neutral} \\
          \hline
          {\textit{Frank Lampard Ghost Goal}} 
           & Germans can't say anything about unsporting behavior.  
           & The 1966 ghost goal had to be paid for. 
           & This was way more clear-cut than 1966. \\
          \hline
          {\textit{Luis Suárez Handball}} 
           & Morality always loses, and nice guys finish last.  
           & Hand of Satan. 
           & He will never step foot in Ghana. \\
          \hline
          {\textit{Maradona Hand of God}} 
           & Number 15 goal is something else...my favourite. Bravo. 
           & The most cheating player in football history. 
           & You will never be able to pick one between Maradona and Messi. \\
          \hline
          {\textit{Jonny Bairstow's Run-Out}} 
           & That’s not cheating, that’s the way of winning. 
           & Same old Aussies, always cheating. 
           & The lesson for the players is ``pay attention''. \\
          \hline
          {\textit{Ashwin Mankading Event}} 
           & If a bowler can keep his foot inside the crease, a batsman can wait with the bat inside the crease until the ball is bowled. What's wrong with that? 
           & If you Mankad, you should be ashamed of yourself. That means you don't have the skill to take the wicket by bowling. 
           & Ashwin merely expressed his disappointment but never wanted a wicket that way. Team decision reflects it. \\
          \hline
          {\textit{The Underarm Incident}} 
           & That time it was legal to bowl underarm according to rules. 
           & That was against the rules!! Couldn't they just ball a normal delivery? I mean there was no way a six would have been surely hit... well there could be... 
           & What were the exact rules for underarm deliveries back then? Were you allowed to bowl as many as you wanted, and if so, why didn't they do it all the time? \\
          \hline
      \end{tabular}
    } % End local grouping
    \caption{Examples of Favor, Against, and Neutral Comments for Controversial Events}
    \label{tab:event_comments}
\end{table*}
%\FloatBarrier % ensures the table is processed before subsequent content





%\vspace{5mm} % Add space before table

% \if 0
% \FloatBarrier % Ensures previous content is finalized before placing this table
% \begin{table*}[!htbp]
%     \centering
%     \begin{adjustbox}{max width=\textwidth}
%         { % Local scope to prevent affecting other tables
%         \renewcommand{\arraystretch}{1.2} % Increase row spacing slightly for better readability
%         \begin{tabularx}{\textwidth}{|c|>{\raggedright\arraybackslash}X|>{\raggedright\arraybackslash}X|>{\raggedright\arraybackslash}X|>{\raggedright\arraybackslash}X|}
%             \hline
%             \textbf{Event} & \textbf{Direct Praise} & \textbf{Indirect Praise} & \textbf{Direct Criticism} & \textbf{Indirect Criticism} \\
%             \hline
%             \multirow{3}{*}{\textit{Maradona Hand of God}}
%              & Maradona is sooo good. 
%              & 13 was simply incredible. 
%              & 3 was handball. Not a goal. A cheat. 
%              & You forgot to add the Hand of God goal. \\[5pt]
%             \cline{2-5}
%              & 13 was simply incredible. 
%              & The legend forever. 
%              & Goal 12 looks like a huge offside. 
%              & Back when players played for the crowd, not money. \\[5pt]
%             \cline{2-5}
%              & The legend forever. 
%              & Back when players played for the crowd, not money. 
%              & Cheating and poor goalkeeping. 
%              & Maradona’s skill ended when he cheated. \\[5pt]
%             \hline
%             \multirow{3}{*}{\textit{Luis Suarez Handball}}
%              & Suarez is a legend... but seeing Asamoah after the match broke my heart. 
%              & Suarez took matters into his own hands. 
%              & Suarez crushed so many African dreams.. absolutely criminal. 
%              & If you watch closely, Suarez wasn't even the only one who tried to handball. \\[5pt]
%             \cline{2-5}
%              & Suarez is a genius. 
%              & He did what he had to do for his country. 
%              & Absolute scumbag play. 
%              & Ghana would have won if it weren't for the handball. \\[5pt]
%             \cline{2-5}
%              & Suarez hero or villain? 
%              & Anyone would have done that though. 
%              & Suarez will forever be the biggest disgrace in modern football. 
%              & It's funny though because both Argentina and Uruguay cheated and both got what they had coming. \\[5pt]
%             \hline
%             \multirow{3}{*}{\textit{Frank Lampard Ghost Goal}}
%              & They were voted the most entertaining team on FIFA.com. 
%              & It took 12 years, but Germany got their karma at last. 
%              & I hate Germany for what they did, it is so sad. 
%              & Wouldn't have made a difference; I'm English, and we would have lost anyway. Germany were by far the better team. \\[5pt]
%             \cline{2-5}
%              & Wouldn't have made a difference; I'm English, and we would have lost anyway. Germany were by far the better team. 
%              & OSCAR-winning performance from football. 
%              & England defended like a bunch of girls. 
%              & With all the technology today, this would never happen now. \\[5pt]
%             \cline{2-5}
%              & My idol was Frank Lampard, and I'm so happy to see, but that referee was an absolute disgrace, blind, and should go to the eye doctor. 
%              & This is the reason why FIFA needs VAR... 
%              & That's just poor by the officials... smh. 
%              & The revenge has been taken. \\[5pt]
%             \hline
%         \end{tabularx}
%         }
%     \end{adjustbox}
%     \caption{Examples of Direct Praise, Indirect Praise, Direct Criticism, and Indirect Criticism}
%     \label{tab:maradona_suarez_lampard_comments}
% \end{table*}
% \FloatBarrier % Ensures subsequent content doesn't move above the table

% \fi

% \begin{table}[htbp]
%     \centering
%     \begin{adjustbox}{max width=\textwidth}
%         \begin{tabularx}{\textwidth}{|c|>{\raggedright\arraybackslash}X|>{\raggedright\arraybackslash}X|>{\raggedright\arraybackslash}X|>{\raggedright\arraybackslash}X|}
%             \hline
%             \textbf{Event} & \textbf{Direct Praise} & \textbf{Indirect Praise} & \textbf{Direct Criticism} & \textbf{Indirect Criticism} \\
%             \hline
%             \multirow{3}{}{\textit{Maradona Hand of God}} 
%              & 1. Maradona is sooo good.  
%              & 1. 13 was simply incredible.  
%              & 1. 3 was handball. Not a goal. A cheat. 
%              & 1. You forgot to add the hand of God goal. \\[5pt]
%             \cline{2-5}
%              & 2. 13 was simply incredible.
%              & 2. The legend forever.
%              & 2. Goal 12 looks like a huge offside.
%              & 2. Back when players played for the crowd, not money. \\[5pt]
%             \cline{2-5}
%              & 3. The legend forever.
%              & 3. Back when players played for the crowd, not money.
%              & 3. Cheating and poor goalkeeping.
%              & 3. Maradona’s skill ended when he cheated. \\
%             \hline
%         \end{tabularx}
%     \end{adjustbox}
%     \caption{Examples of Direct Praise, Indirect Praise, Direct Criticism, and Indirect Criticism for Maradona Hand of God}
%     \label{tab:maradona_comments}
% \end{table}





% \vspace{5mm} % Add space before table

% \begin{table}[htbp]
%     \centering
%     \begin{tabular}{|c|p{12cm}|}
%         \hline
%         \textbf{Event} & \textbf{Example Comments} \\
%         % \hline
%         \multirow{9}{}{\textit{Frank Lampard Ghost Goal}} 
%          & \textbf{Favor} \\
%          & 1. I wonder what Germans think to unsporting behaviour. They don't think. ������ \\
%          & 2. What a disgrace this Manuel is. \\
%          & 3. Could had been one of the greatest games of all time but ref decided, no…. \\
%          \hline
%          & \textbf{Against} \\
%          & 1. Even Geoff Hurst in 1966 said my goal didn’t count even though it was a goal. This is just revenge, I believe. \\
%          & 2. Geoff Hurst's "ghost" winning goal in 1966 had to be paid.  
%            1970 - Losing the QF to Germany after leading 2-0.  
%            1990 - Losing the semi-final on penalties.  
%            2010 - Lampard’s goal not seen by the assistant referee.  
%            The curse started right after the final whistle of the 1966 World Cup Final. \\
%          & 3. Payback for the 1966 final ���� \\
%          \hline
%          & \textbf{Neutral} \\
%          & 1. Inconclusive! Just couldn’t get a good angle on that. \\
%          & 2. Didn’t cross the line. \\
%          & 3. Not enough evidence to overturn the decision. \\
%         \hline
%     \end{tabular}
%     \caption{Example sentiment-based comments for Frank Lampard's Ghost Goal event}
%     \label{tab:frank_lampard_comments}
% \end{table}




% \subsection{Algorithm Details and Rationale}
% The following outlines the overall pipeline and reasoning behind our stance detection approach:
% \begin{itemize}
%     \item \textbf{Pipeline Design:} The pipeline starts with data extraction and preprocessing to ensure the quality of the input comments. Given the noisy nature of social media data, thorough cleaning was essential.
%     \item \textbf{Model Selection:} For the Underarm Incident, the fine-tuned LLaMA-3.1 (Instruct) family of models were chosen for its ability to process long sequences (up to 2048 tokens) and follow the given instructions (as a few shot prompts) to generate coherent responses, making it suitable for detailed stance detection.
%     \item \textbf{Structured Prompts:} We used structured prosmpts to guide the model in classifying comments. This method provided consistent JSON responses, ensuring ease of parsing and reliable extraction of stance labels and reasons.
%     \item \textbf{OLLAMA Framework:} For Jonny Bairstow's and Ashwin's events, the OLLAMA framework allowed for scalable and concurrent processing of comments via API calls. This was critical in handling larger datasets and ensuring a rapid turnaround in analysis.
%     \item \textbf{Evaluation Metrics:} In addition to the stance labels, we compute evaluation metrics such as accuracy, precision, recall, and F1-score to assess model performance.
% \end{itemize}



% Figure~\ref{fig:pipeline} presents a flowchart summarizing the stance detection pipeline.




% % \begin{figure}[ht]
% % \centering
% % \includegraphics[width=0.48\textwidth]{pipeline_flowchart.png}  % Replace with your actual flowchart image
% % \caption{Stance Detection Pipeline Flowchart}
% % \label{fig:pipeline}
% % \end{figure}

% For clarity, the pseudocode in Algorithm~\ref{alg:pipeline} summarizes the pipeline:

% \begin{algorithm}[h]
% \caption{Stance Detection Pipeline}
% \label{alg:pipeline}
% \begin{algorithmic}[1]
% \STATE \textbf{Input:} YouTube comments dataset
% \STATE \textbf{Preprocessing:} Clean comments by removing noise and duplicated data
% \IF{Incident is Underarm}
%     \STATE Use the Unsloth LLaMA-3(Instruct) family of models with a structured prompt
%     \STATE Parse JSON response to extract stance label and reason
% \ELSE
%     \STATE Use the OLLAMA framework with API requests and detailed prompts
%     \STATE Parse JSON response to extract stance label and reason
% \ENDIF
% \STATE \textbf{Output:} Stance labels and evaluation metrics (accuracy, precision, recall, F1-score)
% \end{algorithmic}
% \end{algorithm}



% \vspace{5mm} % Add space before table

\begin{table*}[htbp]
    \centering
    \begin{tabular}{|c|c|c|c|c|c|}
        \hline
        \textbf{Model} & \textbf{Event} & \textbf{Accuracy} & \textbf{Recall (micro)} & \textbf{Precision (micro)} & \textbf{F1 (micro)} \\
         \hline
         DeepSeek-R1-Distill-Llama-8B \\ (Not Fine Tuned) & \textit{Maradona Hand of God} & 22.76\% & 23\% & 31\% & 9\% \\
         & \textit{Luis Suarez Handball} & 33.7\% & 34\% & 49\% & 24\% \\
         & \textit{Frank Lampard Ghost Goal} & 223 \% & 23\% & 31 \% & 9\% \\
         % \hline
          \hline
        DeepSeek-R1-Distill-Llama-8B \\ (Fine Tuned) & \textit{Maradona Hand of God} & 76.20\% & 76\% & 76\% & 76\% \\
         & \textit{Luis Suarez Handball} &  78\% & 78\% & 78 \% & 78\% \\
         & \textit{Frank Lampard Ghost Goal} & 76.2 \% & 76\% & 76 \% &76\% \\
         % \hline
         
        \hline
        Llama 3.1-8b-Instruct (Not Fine Tuned) & \textit{Maradona Hand of God} & 46.8\% & 46\% & 56\% & 40\% \\
         & \textit{Luis Suarez Handball} &  22.8 \% & 23\% & 31 \% & 9\% \\
         & \textit{Frank Lampard Ghost Goal} & 26.3 \% & 26\% & 39 \% & 25\% \\
         \hline
        Llama 3.1-8b-Instruct (Fine Tuned) & \textit{Maradona Hand of God} & 79.04\% & 79\% & 78\% & 77\% \\
         & \textit{Luis Suarez Handball} & 79.5\% & 80\% & 79\% & 79\% \\
         & \textit{Frank Lampard Ghost Goal} & 71.6 \% & 72\% & 72 \% & 72\% \\
        \hline
    \end{tabular}
    \caption{Comparison of models with/without fine-tuning on our constructed dataset}
    \label{tab:student_info}
\end{table*}
\vspace{5mm}





\subsection{Detailed Analysis}


\begin{enumerate}
    \item \textit{Number of samples} (\textbf{Favor}, \textbf{Against}, \textbf{Neutral} and \textbf{Irrelevant})
    \begin{enumerate}
    \item The labels, \textit{Favor} and \textit{Against} is significantly higher for  \textit{Frank Lampard Ghost Goal} compared to other events with \textit{Favor} being comparatively higher, followed by \textit{Luiz Suarez Handball} event.
    \item The \textit number of samples for {Neutral} label is higher for \textit{Luis Suarez Handball} event.
    \item The label \textit{Irrelevant} is significantly higher for \textit{Luis Suarez Handball} event meaning the majority of the comments couldn't be classified into the other three labels.
   
    \end{enumerate}
    \item \textit{Variations of Praise and Criticism}
    \begin{enumerate}
        \item Instances of \textit{Direct Criticism} is highest for \textit{Maradona Hand of God} event.
        \item \textit{Direct Praise} accounts highest for \textit{Luis Suarez Handball} with equal instances for \textit{Maradona Hand of God} and \textit{Frank Lampard Ghost Goal}.
        \item For \textit{Indirect Criticism}, \textit{Frank Lampard Ghost Goal} is highest followed by \textit{Maradona Hand of God}.
        \item In terms of \textit{Favor} label (Direct + Indirect Praise), Frank Lampard event is highest, followed by Maradona and then by Luis Suarez events.
        \item  Similarly, for \textit{Against} label (Direct + Indirect Criticism), Maradona event's count is highest followed by Frank Lampard and then Luis Suarez events.
    
    \end{enumerate}
    
     Overall, \textit{Frank Lampard Ghost Goal} event is highly favoured as well resented by the public. A balance between the three opinions can be found in \textit{Luis Suarez Handball} event.

      \item \textit{Fine Tuning of LLMs}
    \begin{enumerate}
        \item The fine tuning was performed on 80-20 split of the labelled dataset. Without fine-tuning, Llama3.1-8b Instruct got an average of 35.6 \% and with fine-tuning on our constructed dataset, there was a drastic improvement of over 40 \% to 76.71 \%.
        \item This improvement is majorly due to the quality of labels associated with the respective comments after a thorough human verification. Our models can be found here \footnote{\url{https://huggingface.co/YuvrajSingh9886/Llama3.1-8b-Maradona/}} \footnote{\url{https://huggingface.co/YuvrajSingh9886/Llama3.1-8b-Frank-Lampard/}} \footnote{\url{https://huggingface.co/YuvrajSingh9886/Llama-3.1-8b-Luis-Suarez}}. Our entire pipeline can be found here \footnote{\url{https://github.com/YuvrajSingh-mist/Public-Sports-Controversy/tree/master/data/PDFs}}
       
    \end{enumerate}


\item \textit{Indepth analysis of stance labels}
        \begin{enumerate}
            \item We further investigated the primary stance labels, especially \textbf{Favor} and \textbf{Against}, by introducing more granular sub-categories: \textbf{Direct Praise}, \textbf{Indirect Praise}, \textbf{Direct Criticism}, and \textbf{Indirect Criticism}, along with tracking \textbf{Slang Use} and \textbf{Racial Abuse}. This allowed for a finer understanding of how different types of expressions correlate with overall sentiment across the datasets.

            % --- Summary Findings Start Here (as a nested list item) ---
            \item Key relationships observed between these granular sub-categories and the main sentiment labels (referencing data from Tables 4, 5, and 6 \footnote{\url{https://github.com/YuvrajSingh-mist/Public-Sports-Controversy/blob/master/data/PDFs/Tables.pdf}}) include:
                % \begin{itemize} % Start list for each event/individual
                    \item \textbf{Maradona Hand of God Event (Ref: Table 4):}
                        \begin{itemize}
                            \item \textit{Clear Alignment:} Direct Praise (I=1) aligns 100\% with \textbf{Favor} (E=0); Direct Criticism (H=1) aligns 100\% with \textbf{Against} (E=1).
                            \item \textit{Ambiguous Alignment:} Indirect Praise (K=1), Indirect Criticism (J=1), and Slang Use (L=1) are spread across multiple labels.
                            \item \textit{Rare Instance:} Racial Abuse (M=1) appeared infrequently under both \textbf{Favor} and \textbf{Against}.
                        \end{itemize}

                    \item \textbf{Luis Suarez Handball Event (Ref: Table 5):}
                        \begin{itemize}
                            \item \textit{Strong Alignment:} Direct Criticism (I=1) shows 94\% alignment with \textbf{Against} (D=1); Direct Praise (J=1) shows 78\% alignment with \textbf{Favor} (D=0); Racial Abuse (N=1) shows 71\% alignment with \textbf{Against} (D=1).
                            \item \textit{Weak Alignment:} Indirect Criticism (K=1) and Slang Use (M=1) lacked strong correlation with a single label.
                        \end{itemize}

                    \item \textbf{Frank Lampard Ghost Goal Event (Ref: Table 6):}
                         \begin{itemize}
                            \item \textit{Clear Alignment:} Direct Criticism (I) aligns 100\% with \textbf{Against} (H=1); Direct Praise (J) aligns 100\% with \textbf{Favor} (H=0).
                            \item \textit{Notable Trends:} Approx. 54\% of Indirect Criticism (K) comments were labeled \textbf{Favor} (H=0); Approx. 57\% of Slang Use (M) comments were labeled \textbf{Against} (H=1).
                        \end{itemize}
                % \end{itemize} % End list for each event/individual
            % --- Summary Findings End Here ---

        \end{enumerate} % This is the \end{enumerate} before which the content was added
% (End of the outer \item \textit{Indepth analysis...})
    

\end{enumerate}

    
% \end{enumerate}

% \subsection{Challenges Encountered}
% During preprocessing, challenges such as handling noisy data, duplicate entries, and variations in comment formats were encountered. For stance detection, ensuring reliable automated classification and consistent JSON parsing proved difficult. These challenges motivated the use of structured prompts and robust frameworks like OLLAMA.

% \subsection{Future Directions}
% Future work will focus on:
% \begin{itemize}
%   \item Refining sentiment classification models using advanced machine learning techniques.
%   \item Expanding the dataset to include a broader range of sports controversies.
%   \item Enhancing preprocessing methods and fine-tuning model parameters to improve overall performance.
% \end{itemize}

% \section{Ethical Considerations}
% In this study, the ethical use of publicly available YouTube data was ensured. All data were anonymized and processed in accordance with ACM's policies on research involving human subjects. Informed consent was addressed by using publicly accessible data without any direct identification of individuals.

\section{Conclusion}
This study highlights the role of social media in shaping public perception of sports controversies. The integration of automated data extraction and stance detection provides a comprehensive view of audience sentiment. Future enhancements will aim to improve accuracy and broaden the scope of analysis.


\if 0
\section{Template Overview}
As noted in the introduction, the ``\verb|acmart|'' document class can
be used to prepare many different kinds of documentation --- a
double-anonymous initial submission of a full-length technical paper, a
two-page SIGGRAPH Emerging Technologies abstract, a ``camera-ready''
journal article, a SIGCHI Extended Abstract, and more --- all by
selecting the appropriate {\itshape template style} and {\itshape
  template parameters}.

This document will explain the major features of the document
class. For further information, the {\itshape \LaTeX\ User's Guide} is
available from
\url{https://www.acm.org/publications/proceedings-template}.

\subsection{Template Styles}

The primary parameter given to the ``\verb|acmart|'' document class is
the {\itshape template style} which corresponds to the kind of publication
or SIG publishing the work. This parameter is enclosed in square
brackets and is a part of the {\verb|documentclass|} command:
\begin{verbatim}
  \documentclass[STYLE]{acmart}
\end{verbatim}

Journals use one of three template styles. All but three ACM journals
use the {\verb|acmsmall|} template style:
\begin{itemize}
\item {\texttt{acmsmall}}: The default journal template style.
\item {\texttt{acmlarge}}: Used by JOCCH and TAP.
\item {\texttt{acmtog}}: Used by TOG.
\end{itemize}

The majority of conference proceedings documentation will use the {\verb|acmconf|} template style.
\begin{itemize}
\item {\texttt{sigconf}}: The default proceedings template style.
\item{\texttt{sigchi}}: Used for SIGCHI conference articles.
\item{\texttt{sigplan}}: Used for SIGPLAN conference articles.
\end{itemize}

\subsection{Template Parameters}

In addition to specifying the {\itshape template style} to be used in
formatting your work, there are a number of {\itshape template parameters}
which modify some part of the applied template style. A complete list
of these parameters can be found in the {\itshape \LaTeX\ User's Guide.}

Frequently-used parameters, or combinations of parameters, include:
\begin{itemize}
\item {\texttt{anonymous,review}}: Suitable for a ``double-anonymous''
  conference submission. Anonymizes the work and includes line
  numbers. Use with the \texttt{\string\acmSubmissionID} command to print the
  submission's unique ID on each page of the work.
\item{\texttt{authorversion}}: Produces a version of the work suitable
  for posting by the author.
\item{\texttt{screen}}: Produces colored hyperlinks.
\end{itemize}

This document uses the following string as the first command in the
source file:
\begin{verbatim}
\documentclass[sigconf]{acmart}
\end{verbatim}

\section{Modifications}

Modifying the template --- including but not limited to: adjusting
margins, typeface sizes, line spacing, paragraph and list definitions,
and the use of the \verb|\vspace| command to manually adjust the
vertical spacing between elements of your work --- is not allowed.

{\bfseries Your document will be returned to you for revision if
  modifications are discovered.}

\section{Typefaces}

The ``\verb|acmart|'' document class requires the use of the
``Libertine'' typeface family. Your \TeX\ installation should include
this set of packages. Please do not substitute other typefaces. The
``\verb|lmodern|'' and ``\verb|ltimes|'' packages should not be used,
as they will override the built-in typeface families.

\section{Title Information}

The title of your work should use capital letters appropriately -
\url{https://capitalizemytitle.com/} has useful rules for
capitalization. Use the {\verb|title|} command to define the title of
your work. If your work has a subtitle, define it with the
{\verb|subtitle|} command.  Do not insert line breaks in your title.

If your title is lengthy, you must define a short version to be used
in the page headers, to prevent overlapping text. The \verb|title|
command has a ``short title'' parameter:
\begin{verbatim}
  \title[short title]{full title}
\end{verbatim}

\section{Authors and Affiliations}

Each author must be defined separately for accurate metadata
identification.  As an exception, multiple authors may share one
affiliation. Authors' names should not be abbreviated; use full first
names wherever possible. Include authors' e-mail addresses whenever
possible.

Grouping authors' names or e-mail addresses, or providing an ``e-mail
alias,'' as shown below, is not acceptable:
\begin{verbatim}
  \author{Brooke Aster, David Mehldau}
  \email{dave,judy,steve@university.edu}
  \email{firstname.lastname@phillips.org}
\end{verbatim}

The \verb|authornote| and \verb|authornotemark| commands allow a note
to apply to multiple authors --- for example, if the first two authors
of an article contributed equally to the work.

If your author list is lengthy, you must define a shortened version of
the list of authors to be used in the page headers, to prevent
overlapping text. The following command should be placed just after
the last \verb|\author{}| definition:
\begin{verbatim}
  \renewcommand{\shortauthors}{McCartney, et al.}
\end{verbatim}
Omitting this command will force the use of a concatenated list of all
of the authors' names, which may result in overlapping text in the
page headers.

The article template's documentation, available at
\url{https://www.acm.org/publications/proceedings-template}, has a
complete explanation of these commands and tips for their effective
use.

Note that authors' addresses are mandatory for journal articles.

\section{Rights Information}

Authors of any work published by ACM will need to complete a rights
form. Depending on the kind of work, and the rights management choice
made by the author, this may be copyright transfer, permission,
license, or an OA (open access) agreement.

Regardless of the rights management choice, the author will receive a
copy of the completed rights form once it has been submitted. This
form contains \LaTeX\ commands that must be copied into the source
document. When the document source is compiled, these commands and
their parameters add formatted text to several areas of the final
document:
\begin{itemize}
\item the ``ACM Reference Format'' text on the first page.
\item the ``rights management'' text on the first page.
\item the conference information in the page header(s).
\end{itemize}

Rights information is unique to the work; if you are preparing several
works for an event, make sure to use the correct set of commands with
each of the works.

The ACM Reference Format text is required for all articles over one
page in length, and is optional for one-page articles (abstracts).

\section{CCS Concepts and User-Defined Keywords}

Two elements of the ``acmart'' document class provide powerful
taxonomic tools for you to help readers find your work in an online
search.

The ACM Computing Classification System ---
\url{https://www.acm.org/publications/class-2012} --- is a set of
classifiers and concepts that describe the computing
discipline. Authors can select entries from this classification
system, via \url{https://dl.acm.org/ccs/ccs.cfm}, and generate the
commands to be included in the \LaTeX\ source.

User-defined keywords are a comma-separated list of words and phrases
of the authors' choosing, providing a more flexible way of describing
the research being presented.

CCS concepts and user-defined keywords are required for for all
articles over two pages in length, and are optional for one- and
two-page articles (or abstracts).

\section{Sectioning Commands}

Your work should use standard \LaTeX\ sectioning commands:
\verb|\section|, \verb|\subsection|, \verb|\subsubsection|,
\verb|\paragraph|, and \verb|\subparagraph|. The sectioning levels up to
\verb|\subsusection| should be numbered; do not remove the numbering
from the commands.

Simulating a sectioning command by setting the first word or words of
a paragraph in boldface or italicized text is {\bfseries not allowed.}

Below are examples of sectioning commands.

\subsection{Subsection}
\label{sec:subsection}

This is a subsection.

\subsubsection{Subsubsection}
\label{sec:subsubsection}

This is a subsubsection.

\paragraph{Paragraph}

This is a paragraph.

\subparagraph{Subparagraph}

This is a subparagraph.

\section{Tables}

The ``\verb|acmart|'' document class includes the ``\verb|booktabs|''
package --- \url{https://ctan.org/pkg/booktabs} --- for preparing
high-quality tables.

Table captions are placed {\itshape above} the table.

Because tables cannot be split across pages, the best placement for
them is typically the top of the page nearest their initial cite.  To
ensure this proper ``floating'' placement of tables, use the
environment \textbf{table} to enclose the table's contents and the
table caption.  The contents of the table itself must go in the
\textbf{tabular} environment, to be aligned properly in rows and
columns, with the desired horizontal and vertical rules.  Again,
detailed instructions on \textbf{tabular} material are found in the
\textit{\LaTeX\ User's Guide}.

Immediately following this sentence is the point at which
Table~\ref{tab:freq} is included in the input file; compare the
placement of the table here with the table in the printed output of
this document.

\begin{table}
  \caption{Frequency of Special Characters}
  \label{tab:freq}
  \begin{tabular}{ccl}
    \toprule
    Non-English or Math&Frequency&Comments\\
    \midrule
    \O & 1 in 1,000& For Swedish names\\
    $\pi$ & 1 in 5& Common in math\\
    \$ & 4 in 5 & Used in business\\
    $\Psi^2_1$ & 1 in 40,000& Unexplained usage\\
  \bottomrule
\end{tabular}
\end{table}

To set a wider table, which takes up the whole width of the page's
live area, use the environment \textbf{table} to enclose the table's
contents and the table caption.  As with a single-column table, this
wide table will ``float'' to a location deemed more
desirable. Immediately following this sentence is the point at which
Table~\ref{tab:commands} is included in the input file; again, it is
instructive to compare the placement of the table here with the table
in the printed output of this document.

\begin{table}
  \caption{Some Typical Commands}
  \label{tab:commands}
  \begin{tabular}{ccl}
    \toprule
    Command &A Number & Comments\\
    \midrule
    \texttt{{\char'134}author} & 100& Author \\
    \texttt{{\char'134}table}& 300 & For tables\\
    \texttt{{\char'134}table}& 400& For wider tables\\
    \bottomrule
  \end{tabular}
\end{table}

Always use midrule to separate table header rows from data rows, and
use it only for this purpose. This enables assistive technologies to
recognise table headers and support their users in navigating tables
more easily.

\section{Math Equations}
You may want to display math equations in three distinct styles:
inline, numbered or non-numbered display.  Each of the three are
discussed in the next sections.

\subsection{Inline (In-text) Equations}
A formula that appears in the running text is called an inline or
in-text formula.  It is produced by the \textbf{math} environment,
which can be invoked with the usual
\texttt{{\char'134}begin\,\ldots{\char'134}end} construction or with
the short form \texttt{\$\,\ldots\$}. You can use any of the symbols
and structures, from $\alpha$ to $\omega$, available in
\LaTeX~\cite{Lamport:LaTeX}; this section will simply show a few
examples of in-text equations in context. Notice how this equation:
\begin{math}
  \lim_{n\rightarrow \infty}x=0
\end{math},
set here in in-line math style, looks slightly different when
set in display style.  (See next section).

\subsection{Display Equations}
A numbered display equation---one set off by vertical space from the
text and centered horizontally---is produced by the \textbf{equation}
environment. An unnumbered display equation is produced by the
\textbf{displaymath} environment.

Again, in either environment, you can use any of the symbols and
structures available in \LaTeX\@; this section will just give a couple
of examples of display equations in context.  First, consider the
equation, shown as an inline equation above:
\begin{equation}
  \lim_{n\rightarrow \infty}x=0
\end{equation}
Notice how it is formatted somewhat differently in
the \textbf{displaymath}
environment.  Now, we'll enter an unnumbered equation:
\begin{displaymath}
  \sum_{i=0}^{\infty} x + 1
\end{displaymath}
and follow it with another numbered equation:
\begin{equation}
  \sum_{i=0}^{\infty}x_i=\int_{0}^{\pi+2} f
\end{equation}
just to demonstrate \LaTeX's able handling of numbering.

\section{Figures}

The ``\verb|figure|'' environment should be used for figures. One or
more images can be placed within a figure. If your figure contains
third-party material, you must clearly identify it as such, as shown
in the example below.
\begin{figure}[h]
  \centering
  \includegraphics[width=\linewidth]{sample-franklin}
  \caption{1907 Franklin Model D roadster. Photograph by Harris \&
    Ewing, Inc. [Public domain], via Wikimedia
    Commons. (\url{https://goo.gl/VLCRBB}).}
  \Description{A woman and a girl in white dresses sit in an open car.}
\end{figure}

Your figures should contain a caption which describes the figure to
the reader.

Figure captions are placed {\itshape below} the figure.

Every figure should also have a figure description unless it is purely
decorative. These descriptions convey what’s in the image to someone
who cannot see it. They are also used by search engine crawlers for
indexing images, and when images cannot be loaded.

A figure description must be unformatted plain text less than 2000
characters long (including spaces).  {\bfseries Figure descriptions
  should not repeat the figure caption – their purpose is to capture
  important information that is not already provided in the caption or
  the main text of the paper.} For figures that convey important and
complex new information, a short text description may not be
adequate. More complex alternative descriptions can be placed in an
appendix and referenced in a short figure description. For example,
provide a data table capturing the information in a bar chart, or a
structured list representing a graph.  For additional information
regarding how best to write figure descriptions and why doing this is
so important, please see
\url{https://www.acm.org/publications/taps/describing-figures/}.

\subsection{The ``Teaser Figure''}

A ``teaser figure'' is an image, or set of images in one figure, that
are placed after all author and affiliation information, and before
the body of the article, spanning the page. If you wish to have such a
figure in your article, place the command immediately before the
\verb|\maketitle| command:
\begin{verbatim}
  \begin{teaserfigure}
    \includegraphics[width=\textwidth]{sampleteaser}
    \caption{figure caption}
    \Description{figure description}
  \end{teaserfigure}
\end{verbatim}

\section{Citations and Bibliographies}

The use of \BibTeX\ for the preparation and formatting of one's
references is strongly recommended. Authors' names should be complete
--- use full first names (``Donald E. Knuth'') not initials
(``D. E. Knuth'') --- and the salient identifying features of a
reference should be included: title, year, volume, number, pages,
article DOI, etc.

The bibliography is included in your source document with these two
commands, placed just before the \verb|\end{document}| command:
\begin{verbatim}
  \bibliographystyle{ACM-Reference-Format}
  \bibliography{bibfile}
\end{verbatim}
where ``\verb|bibfile|'' is the name, without the ``\verb|.bib|''
suffix, of the \BibTeX\ file.

Citations and references are numbered by default. A small number of
ACM publications have citations and references formatted in the
``author year'' style; for these exceptions, please include this
command in the {\bfseries preamble} (before the command
``\verb|\begin{document}|'') of your \LaTeX\ source:
\begin{verbatim}
  \citestyle{acmauthoryear}
\end{verbatim}


  Some examples.  A paginated journal article \cite{Abril07}, an
  enumerated journal article \cite{Cohen07}, a reference to an entire
  issue \cite{JCohen96}, a monograph (whole book) \cite{Kosiur01}, a
  monograph/whole book in a series (see 2a in spec. document)
  \cite{Harel79}, a divisible-book such as an anthology or compilation
  \cite{Editor00} followed by the same example, however we only output
  the series if the volume number is given \cite{Editor00a} (so
  Editor00a's series should NOT be present since it has no vol. no.),
  a chapter in a divisible book \cite{Spector90}, a chapter in a
  divisible book in a series \cite{Douglass98}, a multi-volume work as
  book \cite{Knuth97}, a couple of articles in a proceedings (of a
  conference, symposium, workshop for example) (paginated proceedings
  article) \cite{Andler79, Hagerup1993}, a proceedings article with
  all possible elements \cite{Smith10}, an example of an enumerated
  proceedings article \cite{VanGundy07}, an informally published work
  \cite{Harel78}, a couple of preprints \cite{Bornmann2019,
    AnzarootPBM14}, a doctoral dissertation \cite{Clarkson85}, a
  master's thesis: \cite{anisi03}, an online document / world wide web
  resource \cite{Thornburg01, Ablamowicz07, Poker06}, a video game
  (Case 1) \cite{Obama08} and (Case 2) \cite{Novak03} and \cite{Lee05}
  and (Case 3) a patent \cite{JoeScientist001}, work accepted for
  publication \cite{rous08}, 'YYYYb'-test for prolific author
  \cite{SaeediMEJ10} and \cite{SaeediJETC10}. Other cites might
  contain 'duplicate' DOI and URLs (some SIAM articles)
  \cite{Kirschmer:2010:AEI:1958016.1958018}. Boris / Barbara Beeton:
  multi-volume works as books \cite{MR781536} and \cite{MR781537}. A
  couple of citations with DOIs:
  \cite{2004:ITE:1009386.1010128,Kirschmer:2010:AEI:1958016.1958018}. Online
  citations: \cite{TUGInstmem, Thornburg01, CTANacmart}.
  Artifacts: \cite{R} and \cite{UMassCitations}.

\section{Acknowledgments}

Identification of funding sources and other support, and thanks to
individuals and groups that assisted in the research and the
preparation of the work should be included in an acknowledgment
section, which is placed just before the reference section in your
document.

This section has a special environment:
\begin{verbatim}
  \begin{acks}
  ...
  \end{acks}
\end{verbatim}
so that the information contained therein can be more easily collected
during the article metadata extraction phase, and to ensure
consistency in the spelling of the section heading.

Authors should not prepare this section as a numbered or unnumbered {\verb|\section|}; please use the ``{\verb|acks|}'' environment.

\section{Appendices}

If your work needs an appendix, add it before the
``\verb|\end{document}|'' command at the conclusion of your source
document.

Start the appendix with the ``\verb|appendix|'' command:
\begin{verbatim}
  \appendix
\end{verbatim}
and note that in the appendix, sections are lettered, not
numbered. This document has two appendices, demonstrating the section
and subsection identification method.

\section{Multi-language papers}

Papers may be written in languages other than English or include
titles, subtitles, keywords and abstracts in different languages (as a
rule, a paper in a language other than English should include an
English title and an English abstract).  Use \verb|language=...| for
every language used in the paper.  The last language indicated is the
main language of the paper.  For example, a French paper with
additional titles and abstracts in English and German may start with
the following command
\begin{verbatim}
\documentclass[sigconf, language=english, language=german,
               language=french]{acmart}
\end{verbatim}

The title, subtitle, keywords and abstract will be typeset in the main
language of the paper.  The commands \verb|\translatedXXX|, \verb|XXX|
begin title, subtitle and keywords, can be used to set these elements
in the other languages.  The environment \verb|translatedabstract| is
used to set the translation of the abstract.  These commands and
environment have a mandatory first argument: the language of the
second argument.  See \verb|sample-sigconf-i13n.tex| file for examples
of their usage.

\section{SIGCHI Extended Abstracts}

The ``\verb|sigchi-a|'' template style (available only in \LaTeX\ and
not in Word) produces a landscape-orientation formatted article, with
a wide left margin. Three environments are available for use with the
``\verb|sigchi-a|'' template style, and produce formatted output in
the margin:
\begin{description}
\item[\texttt{sidebar}:]  Place formatted text in the margin.
\item[\texttt{marginfigure}:] Place a figure in the margin.
\item[\texttt{margintable}:] Place a table in the margin.
\end{description}

%%
%% The acknowledgments section is defined using the "acks" environment
%% (and NOT an unnumbered section). This ensures the proper
%% identification of the section in the article metadata, and the
%% consistent spelling of the heading.
\begin{acks}
To Robert, for the bagels and explaining CMYK and color spaces.
\end{acks}
\fi
%%
%% The next two lines define the bibliography style to be used, and
%% the bibliography file.
\bibliographystyle{ACM-Reference-Format}
\bibliography{sample-base}


%%
%% If your work has an appendix, this is the place to put it.

\if 0
\appendix

\section{Research Methods}

\subsection{Part One}

Lorem ipsum dolor sit amet, consectetur adipiscing elit. Morbi
malesuada, quam in pulvinar varius, metus nunc fermentum urna, id
sollicitudin purus odio sit amet enim. Aliquam ullamcorper eu ipsum
vel mollis. Curabitur quis dictum nisl. Phasellus vel semper risus, et
lacinia dolor. Integer ultricies commodo sem nec semper.

\subsection{Part Two}

Etiam commodo feugiat nisl pulvinar pellentesque. Etiam auctor sodales
ligula, non varius nibh pulvinar semper. Suspendisse nec lectus non
ipsum convallis congue hendrerit vitae sapien. Donec at laoreet
eros. Vivamus non purus placerat, scelerisque diam eu, cursus
ante. Etiam aliquam tortor auctor efficitur mattis.

\section{Online Resources}

Nam id fermentum dui. Suspendisse sagittis tortor a nulla mollis, in
pulvinar ex pretium. Sed interdum orci quis metus euismod, et sagittis
enim maximus. Vestibulum gravida massa ut felis suscipit
congue. Quisque mattis elit a risus ultrices commodo venenatis eget
dui. Etiam sagittis eleifend elementum.

Nam interdum magna at lectus dignissim, ac dignissim lorem
rhoncus. Maecenas eu arcu ac neque placerat aliquam. Nunc pulvinar
massa et mattis lacinia.
\fi
\end{document}
\endinput
%%
%% End of file `sample-sigconf.tex'.
